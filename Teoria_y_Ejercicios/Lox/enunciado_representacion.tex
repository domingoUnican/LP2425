% Created 2025-03-04 mar 10:02
% Intended LaTeX compiler: pdflatex
\input{$HOME/Git_Repositories/encargos/ALGT/hojas/source/cabeza}
\author{Helmut Krauser}
\date{\today}
\title{}
\begin{document}

\vspace{5mm}\centerline{\large\bf Representación de Código}\vspace{5mm}
\label{sec:orgc66f9a9}

Para poder trabajar con código Lox de una manera eficiente, es necesario poder 
tener una representación de este código dentro del lenguaje donde escribamos un compilador.

El código del lenguaje Lox tiene una estructura, que está definida por una \href{https://craftinginterpreters.com/appendix-i.html}{gramática libre de contexto}.
Nuestro objetivo en está práctica es preparar una representación del código de Lox dentro del fichero \emph{Representacion.py}.

Como Python permite la programación orientada a objetos, vamos a utilizar clases para representar código Lox.
Demos un extracto del código que debemos generar:
\begin{minted}[frame=lines,fontsize=\scriptsize,linenos=]{python}
from dataclasses import dataclass
from typing import List 

@dataclass
class Program:
    declarations: List[Declaration]

@dataclass
class Declaration:
    pass

@dataclass
class ClassDeclaration(Declaration):
    name: str
    father: str
    methods: List[Function]

@dataclass
class FunctionDeclaration(Declaration):
    fun: Function

@dataclass
class VarDeclaration(Declaration):
    name: str
    expr: Expression

@dataclass
class statement(Declaration):
    pass

@dataclass
class Function:
    name: str
    params: List[Parameter]
    body: Block

\end{minted}
\end{document}
